O seu documento deve ter uma conclusão. Nesse capítulo  você deve explicitar o que se pode concluir a partir dos resultados obtidos como o trabalho realizado. 

No caso desse documento uma conclusão esperada é que voce tenha lido este texto e aprendido, pelos exemplos, como preparar seu documento. Lembre que caso o seu documento seja uma dissertação ou uma tese, será disponibilizado para consulta pública pela CAPES. 

Considerando que as informações aqui fornecidas sobre edição de documentos usando \TeX \ e \LaTeX \ são muito resumidas, recomenda-se que você consulte livros especializados em \TeX \ e \LaTeX \ tais como o livro do \TeX \ escrito por Donald E. Knuth \cite{knuth:tex} ou o livro do \LaTeX \ escrito por Leslie Lamport \cite{lamport:latex}, para suas futuras dúvidas.
\abreviatura{CAPES}{Coordenação de Aperfeiçoamento de Pessoal de N\'{i}vel Superior}
\abreviatura{CNPq}{Conselho Nacional de Desenvolvimento Científico e Tecnológico}
\abreviatura{FINEP}{Financiadora de Estudos e Projetos}
\abreviatura{BNDES}{Banco Nacional de Desenvolvimento Econômico e Social}
\abreviatura{FIESP}{Federação das Indústrias do Estado de São Paulo}