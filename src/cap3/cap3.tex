%TCIDATA{LaTeXparent=0,0,main.tex}

\chapter{Citações e referências}

De modo geral é necessário fazer referência a trabalhos anteriores
que serviram de base para o desenvolvimento de seu trabalho. As
citações e a apresentação das referências no final do
documento deve seguir o padrão ABNT que pode ser implementado com o pacote

\texttt{abntex2cite}.

\noindent A lista dos trabalhos que são referenciados neste
exemplo é armazenada no arquivo \newline\texttt{src/bib/main.bib}.

\section{Referenciando}

Veja os exemplos: O programa \TeX {}\ foi idealizado e implementado
inicialmente na decada de 70 \cite{knuth:tex}. \LaTeX {}\ \cite{lamport:latex}
é uma das várias extensões do \TeX \ \cite{knuth:tex}. Há
várias implementações de \LaTeX {}\ \cite{lamport:latex} para PCs
\cite{furuta:pctex}.

\section{BibTeX}

Lembre-se de executar o BibTeX\ \cite{patashnik:bibtex} e, em seguida, executar
o \LaTeX \ \cite{lamport:latex} duas vezes consecutivas, sempre que novas
citações forem incluídas no texto.

\section{Referências}
 \lipsum[1-3]